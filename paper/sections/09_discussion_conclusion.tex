\section{Discussion and conclusion: what SBT predicts about space}
\label{sec:discussion}

This paper answers the title question in SBT terms: \emph{to plot a stone} is not to recover coordinates from a pre-existing container, but to construct a stable notion of location, distance, and transport from finite-interface closure.  When packaging, staging, and accounting cohere under refinement, a space-like layer becomes available.  When they do not, the diagnostics indicate precisely which aspect of closure fails.

\subsection{What we showed}
We implemented and audited a concrete emergence pipeline from micro-dynamics to a macro metric, then tested it across multiple substrates.  The central empirical claims are:

\begin{itemize}
  \item \textbf{Points are packaged states.}  A ``point'' at a given resolution is an equivalence class induced by a lens $f:Z\to X$ (P5).  It is not assumed; it is constructed.
  \item \textbf{Distance is optimized accounting.}  From the induced macro kernel $\widehat{P}=UP^\tau C$ we define costs from negative log likelihood and distances as minimal protocol cost (shortest paths).  This realizes P6 (ledger) optimized over P3 (composition of moves).
  \item \textbf{Geometry is coherence under refinement.}  A single distance matrix is not yet geometry; geometry is a claim that the packaged layer stabilizes across staging and a refinement ladder.  We operationalize this via closure idempotence and prototype stability defects, route mismatch, inter-scale distortion (after rescaling), and connectivity.
  \item \textbf{Curvature appears as holonomy.}  Distance coherence alone does not detect curvature.  Curvature appears as a stable loop residue of local transport (P3), and we empirically separate plane-like and sphere-like substrates by a holonomy distribution shift.
  \item \textbf{The Pythagorean form emerges in an isotropic diffusion regime.}  Under staged isotropic diffusion, negative log transition probability becomes approximately quadratic and separable, and the Pythagorean residual collapses, while a Manhattan control fails as expected.
\end{itemize}

Taken together, these results support the SBT thesis stated in the introduction: \emph{geometry is what you get when repeated packaging produces a stable notion of nearby, composition of moves, and cost of moving, and that structure closes under refinement}.  Importantly, the paper is not a collection of isolated plots: it is a demonstration that SBT's closure primitives can be made computationally concrete and audited in a way that distinguishes regimes (flat versus curved versus fractal versus constrained deformation).

\subsection{What we did not claim}
The results are deliberately scoped.  In particular:

\begin{itemize}
  \item \textbf{We did not claim that geometry is fundamental.}  The Euclidean and Pythagorean structures observed here are not asserted to be axioms of reality; they are shown to be stable higher-layer accounting laws under specific staging, isotropy, and packaging choices.
  \item \textbf{We did not claim uniqueness.}  Different lens families (different packaging strategies) can induce different macro points and therefore different geometries.  This is expected in SBT: geometry is layer-relative.
  \item \textbf{We did not prove asymptotic limits.}  All constructions are finite and operational.  ``Limit behavior'' is inferred from coherence across ladders, not from a formal $\varepsilon\to 0$ theorem.
  \item \textbf{We did not reduce curvature to a single scalar.}  Holonomy is a diagnostic built from local embeddings and alignments; it is a robust separator under canonical parameters, but it remains sensitive to neighborhood choices, as documented in the robustness sweeps.
\end{itemize}

These non-claims are not disclaimers; they are part of the SBT stance.  A space-like layer is a conditional birth, and its claims should be phrased in terms of stability under the very closure operations that define it.

\subsection{Predictions and next experiments}
The six-birds framing makes concrete predictions about when different geometric regimes should appear.

\paragraph{Predictions.}
\begin{itemize}
  \item \textbf{Constraints shape geometry (P2).}  Changing feasibility (gating, anisotropy, locality) should predictably deform the induced metric and transport invariants.  In particular, strong directional constraints should produce anisotropic geodesics and may induce effective ``cones'' of reachability in the macro space.
  \item \textbf{Curvature is protocol noncommutativity (P3).}  Whenever local transports are forced to be patchwise (because the layer is packaged), noncommutativity should generically appear, and holonomy-like residues should persist under refinement when the layer is coherent.
  \item \textbf{Smooth versus fractal is a refinement-regime question (P4/P5/P6).}  Away from critical regimes, refinement should tend to smooth local neighborhoods toward Euclidean-like behavior.  Near critical or self-similar regimes, refinement should stabilize scale laws (non-integer dimensions, anomalous diffusion) rather than smooth tangent structure.  We use ``fixed point'' language only as an analogy: we do not implement a formal renormalization operator or prove convergence.
  \item \textbf{Pythagoras is an accounting identity, not an axiom (P6/P4/P2).}  Quadratic-additive structure should appear precisely in regimes where staged dynamics yields approximately Gaussian displacement statistics with isotropy; it should fail under anisotropy, strong constraints, or finite-size aliasing.
\end{itemize}

\paragraph{Next experiments (concrete).}
Several natural extensions would strengthen the emergence calculus and broaden applicability:
\begin{itemize}
  \item \textbf{Operator emergence beyond distance (P1).}  Study when induced macro operators become local and stable (discrete Laplacians, gradient-like operators, connections) and whether curvature can be recovered in multiple equivalent ways (e.g., via commutators of induced operators).
  \item \textbf{Learned lenses and prototypes.}  Instead of fixing a lens ladder by diffusion clustering, learn packaging and prototype choices by explicitly minimizing closure defects and distortion across scales, turning ``geometry discovery'' into an optimization problem.
  \item \textbf{Alternative ledgers (P6).}  Replace negative log likelihood with other accounting notions (control energy, communication cost, repair cost) and test which ledgers yield which emergent geometries on the same substrate.
  \item \textbf{Beyond synthetic substrates.}  Apply the pipeline to real-world substrates that are naturally graph-like (transport networks, biological connectomes, interaction graphs) to test whether the same coherence diagnostics predict when a usable geometry layer appears.
  \item \textbf{Formal convergence targets.}  Connect the empirical coherence criteria to formal notions of convergence of metric spaces (e.g., Gromov--Hausdorff style) and to renormalization-style fixed points for the lens ladder \cite{Gromov1999Metric}.
\end{itemize}

\subsection{Six birds, one end-to-end pipeline}
It is worth restating the complete pipeline in primitive terms, because it is the core computational contribution.

\begin{itemize}
  \item \textbf{P5 Packaging} creates the candidate ``points'' as equivalence classes under indistinguishability: $f:Z\to X$.
  \item \textbf{P4 Staging} provides the ladder (both in $\tau$ and in refinement levels) on which stability can be tested and on which smooth versus fractal regimes separate.
  \item \textbf{P1 Operator rewrite} is realized by closure: inserting prototypes to rewrite micro dynamics as a macro kernel $\widehat{P}=UP^\tau C$ and auditing its idempotence and stability.
  \item \textbf{P6 Accounting} turns macro feasibility into a ledger (cost), from which distance is defined.
  \item \textbf{P3 Protocols} enter twice: distance is the minimal cost over composed move protocols (shortest paths), and curvature appears as the residue of composing local transports around loops (holonomy).
  \item \textbf{P2 Constraints} shape the feasible protocol space, thereby deforming the geometry and potentially changing which fixed point the refinement ladder approaches.
\end{itemize}

\paragraph{Conclusion.}
Plotting is therefore not a primitive act; it is a successful closure.  When closure succeeds, we obtain geometry as a stable higher-layer description of what moves are possible and what they cost.  When closure fails, the diagnostics tell us why.  In this sense, \emph{space is not where the stone is}; space is what becomes true about the stone when the six birds make a stable map possible.

\section*{Declarations}

\paragraph{Corresponding author.}
Correspondence to Ioannis Tsiokos (\texttt{ioannis@automorph.io}).

\paragraph{Competing interests.}
The author declares no competing interests.

\paragraph{Funding.}
No external funding was received for this research.

\paragraph{Ethics approval and consent to participate.}
Not applicable; this study involves computational experiments only and uses no human participants, animal subjects, or personal data.

\paragraph{Data availability.}
All generated artifacts (JSON and CSV files) are available in the repository and in the archived release. No external datasets were used; all data are produced by the included scripts.

\paragraph{Code availability.}
Source code is available at \url{https://github.com/ioannist/six-birds-space}. A permanent archive of the submission version is deposited at Zenodo under DOI \href{https://doi.org/10.5281/zenodo.18494975}{10.5281/zenodo.18494975}.

\paragraph{Author contributions.}
I.T.\ is the sole author and was responsible for conceptualization, methodology, software development, formal analysis, writing, and visualization.

\paragraph{Use of AI/LLMs.}
LLM tools (Claude, Anthropic) were used as coding assistants for software scaffolding and manuscript formatting. All scientific content, claims, and experimental design were produced by the author. LLM outputs were reviewed and validated before inclusion.
